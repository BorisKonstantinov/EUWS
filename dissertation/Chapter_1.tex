

Western Europe is subject to various natural disasters. Of these, the most damage is accumulated from windstorms caused by extratropical cyclones. Windstorms consist of severe winds, which are able to move loose shingles from rooftops and snap twigs from trees at their weakest or uproot entire trees and cause devastating structural damage at their most severe. As identified from the Copernicus Extreme Wind catalogue, in the period 1990 to 2012, the 20 most severe storms have caused more than 58 billion dollars worth of insurance loss (indexed to 2012) alone. Understanding windstorms and improving our ability to forecast that accurately and reliably is therefore imperative for both the public and private sector.

At present, the (re)insurance industry utilises statistical analysis of the relationship between windstorms and environmental factors that affect them, such as seasonal effects and the state of the North Atlantic Oscillation. This is prefferential compared to running Earth Climate Models like the Met Office Global Seasonal Forecasting System 5, which requires a non-trivial amount of computational power, which is expensive. Statistical analysis can provide information on the expected frequency and severity of windstorms on a subdecadal timescale, without necessitating the expensive process of producing a forecast. The analysis often includes a quantifiable measure of storm severity known as a Storm Severity Index (SSI). These can be modified to better suit the purpose of a study, and as such there is no one best SSI, as each is defined by the parameters a researcher wishes to investigate.
In this project, we propose an SSI which focusses on computing the energy exerted by the flow of air near the surface, with the goal of gaining the ability to consider all parameters of wind for every point within a windstorm and thus allowing for higher precision in results.

Industry is limited in terms of reliable data on windstorms to roughly the last 70 years and base their climatological averages on a span of 20 years. In this study, we aim to investigate whether the choice of period biases the obtained statistical relationships between windstorms and the NAO, and what effect a shorter or longer dataset will have on results.

Windstorms associated with a positive or negative NAO state are well studied as these are the states when the NAO has the greatest impact on the behaviour of extratropical cyclones in the North Atlantic; however, less research is done on windstorms observed during a neutral NAO state. This is the second problem investigated in this study, aiming to provide a measure of the return period of NAO neutral windstorms. 
Lastly, we investigate the trajectories followed by windstorms during different NAO states, attempting to identify areas with a high risk of windstorm damage.